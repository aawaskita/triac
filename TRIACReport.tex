\documentclass[a4paper,11pt]{report}
\usepackage[T1]{fontenc}
\usepackage[utf8]{inputenc}
\usepackage{lmodern,url}
\usepackage{graphicx}
\usepackage{hyperref}
\usepackage{pslatex}
\usepackage{listings}
\usepackage{textcomp}
\usepackage{float}
\usepackage[paper=a4paper,headheight=0pt,left=4cm,top=3cm,right=3cm,bottom=3cm]{geometry}
\usepackage{titling}
\usepackage{pdfpages}
\newcommand{\subtitle}[1]{%
  \posttitle{%
    \par\end{center}
    \begin{center}\large#1\end{center}
    \vskip0.5em}%
}
\newcommand{\addChapter}[1]{\phantomsection \addcontentsline{toc}{chapter}{#1}}
% Tambahkan berkas PDF ke dalam laporan dan gunakan style laporan  
% terhadap berkas ini. 
\newcommand{\inpdf}[1]{
	\includepdf[pages=-,pagecommand={\thispagestyle{fancy}}]{#1.pdf}}
% 
% Tambahkan berkas PDF ke dalam laporan. 
\newcommand{\putpdf}[1]{\includepdf[pages=-]{#1.pdf}}
% 
\include{hype.indonesia}

\renewcommand{\contentsname}{Daftar Isi}
\renewcommand{\chaptername}{BAB}
\renewcommand{\bibname}{Daftar Referensi}
\renewcommand{\listfigurename}{Daftar Gambar}
\renewcommand\lstlistlistingname{Daftar Program}
\renewcommand{\figurename}{Gambar}
%\title{Lampiran II}
\title{Kajian Komputasi Dinamika Fluida berbasis OpenFOAM}
\author{Arya Adhyaksa Waskita}
\date{January 31, 2017}
\begin{document}
\include{sampul}
%\tableofcontents

\pagenumbering{roman}
%\maketitle
\clearpage
\setcounter{page}{2}
\addChapter{Daftar Gambar}
\tableofcontents
%\clearpage
\listoffigures
\addChapter{Daftar Program}
\lstlistoflistings
%\clearpage
\pagenumbering{arabic}

\chapter{Pendahuluan}
BATAN saat ini tengah berencana membangun reaktor riset baru berbasis HTGR (\textit{High Temperature Gas-cooled Reactor}) \cite{wang2004integrated} sebagai persiapan PLTN, yang akan dibangun di Indonesia di masa depan \cite{rde}. Salah satu yang perlu diperhatikan dalam pengembangan reaktor jenis ini adalah bahan bakarnya yang berjenis \textit{pebble} yang bentuknya dapat diilustrasikan seperti pada Gambar \ref{fig:bentukpebble}. Bahan bakar harus dirancang sedemikian rupa sehingga rasio gagalnya bahan bakar selama operasi minimal. 

\begin{figure}[h]
  \centering
  \includegraphics[scale=.5]{pics/triso1.png}
  \caption[Ilustrasi bentuk bahan bakar \textit{pebble}]{Ilustrasi bentuk bahan bakar \textit{pebble} \cite{wang2004integrated}}
  \label{fig:bentukpebble}
\end{figure} 

Bahan bakar berjenis \textit{pebble} ini memiliki komponen utama yang dalam Gambar \ref{fig:bentukpebble} disebut sebagai \textit{coated particle}. Komposisi elemen pelapis (\textit{coated}) dapat diilustrasikan dalam Gambar \ref{fig:pelapis}. Dalam upaya 

\begin{figure}[h]
  \centering
  \includegraphics[scale=.5]{pics/triso.png}
  \caption[Komposisi elemen pelapis partikel]{Komposisi elemen pelapis partikel \cite{wang2004integrated}}
  \label{fig:pelapis}
\end{figure}

\chapter{Alur Perhitungan}
\section{Pendahuluan}
Secara umum, perhitungan TRIAC mengikuti diagram alir seperti pada Gambar \ref{fig:flowchart} berikut. Sementara kode sumbernya disajikan dalam Listing \ref{triac.py}.
\begin{figure}[h]
  \centering
  \includegraphics[scale=.5]{pics/Flowchart.png}
  \caption{Diagram alir perhitungan TRAIC}
  \label{fig:flowchart}
\end{figure}

\vfill
\scriptsize
\lstinputlisting[language=python, numbers=left, numberstyle=\tiny, caption=triac.py, showstringspaces=false, label=triac.py]{../TRIAC/inputdata.py}
\normalsize

\section{Membaca \textit{file input}}
Sub rutin ini ditujukan untuk membaca file input dengan format seperti terdapat pada Lampiran 1. Sub rutin ini menggunakan skema yang kaku karena identifikasi nilai-nilai yang akan dibaca ditentukan oleh suatu teks tertentu. Setelah teks yang menjadi penanda, nilai-nilai yang dibutuhkan dibaca. Tetapi, nilai tersebut dapat langsung berada dalam satu baris bersama dengan teks penanda, atau berada pada baris yang berbeda. Sub rutin ini terdapat pada baris ke-3 s/d baris ke-69 dalam Listing \ref{triac.py}

Terdapat empat jenis data yang perlu dibaca dari \textit{file input} dalam Lampiran 1, masing-masing adalah sebagai berikut.
\begin{enumerate}
  \item Data tentang geometri \textit{pebble}. Data ini diidentifikasi menggunakan teks yang didefinisikan oleh variabel \texttt{statusGeometry} (baris ke-5 pada Listing \ref{triac.py}). Di dalam data geometri, terdapat empat data berbeda, masing-masing secara berurutan adalah panjang jejari \textit{pebble} terluar, OPyC (\textit{Outer Pyrolitic Carbon}), SiC (\textit{Silicon Carbide}), IPyC (\textit{Inner Pyrolitic Carbon}), \textit{buffer} dan kernel. 
  \item Data tentang kekuatan SiC. Data ini diidentifikasi menggunakan teks yang didefinisikan oleh varibel \texttt{statusCharacteristics} (baris ke-6 pada Listing \ref{triac.py}). Ada empat nilai yang perlu dibaca terkait kekuatan SiC, masing-masing adalah SiC \textit{Tensile Strength} [Pa],	\textit{Weibull Modulus	Burnup} [FIMA],	\textit{Fission Yield of stable fission gasses} [Ff],	\textit{Fast Neutron Fluence}	dan rasio berat Th terhadap U-235 pada kernel.
    \item Data tentang sejarah irradiasi. Data ini diidentifikasi menggunakan teks yang didefinisikan oleh variabel \texttt{statusIrradiation} (baris ke-) Listing \ref{triac.py}. Data ini merupakan data temperatur bahan bakar \textit{pebble} pada selang waktu tertentu. Sebagai contoh, data yang disajikan pada Lampiran 1 diambil pada selang waktu 17 hari.
\end{enumerate}.
 
% Daftar Pustaka
\bibliographystyle{IEEEtran}
\bibliography{report}

\begin{appendix}
	\include{markLampiran}
	\setcounter{page}{2}
	%-----------------------------------------------------------------------------%
\addChapter{Lampiran 1}
\chapter*{Lampiran 1}
%-----------------------------------------------------------------------------%

\includepdf{inputExample1}
\includepdf{inputExample2}
%\label{lamp:inputExample}

\end{appendix}

\end{document}
