\documentclass[a4paper,11pt]{report}
\usepackage[T1]{fontenc}
\usepackage[utf8]{inputenc}
\usepackage{lmodern,url}
\usepackage{graphicx}
\usepackage{hyperref}
\usepackage{pslatex}
\usepackage{listings}
\usepackage{textcomp}
\usepackage{float}
\usepackage[paper=a4paper,headheight=0pt,left=4cm,top=3cm,right=3cm,bottom=3cm]{geometry}
\usepackage{titling}
\usepackage{pdfpages}
\usepackage{booktabs}
\usepackage[version=4]{mhchem}
\usepackage{isotope}
\newcommand{\subtitle}[1]{%
  \posttitle{%
    \par\end{center}
    \begin{center}\large#1\end{center}
    \vskip0.5em}%
}
\newcommand{\ra}[1]{\renewcommand{\arraystretch}{#1}}
\newcommand{\addChapter}[1]{\phantomsection \addcontentsline{toc}{chapter}{#1}}
% Tambahkan berkas PDF ke dalam laporan dan gunakan style laporan  
% terhadap berkas ini. 
\newcommand{\inpdf}[1]{
	\includepdf[pages=-,pagecommand={\thispagestyle{fancy}}]{#1.pdf}}
% 
% Tambahkan berkas PDF ke dalam laporan. 
\newcommand{\putpdf}[1]{\includepdf[pages=-]{#1.pdf}}
\renewcommand*\descriptionlabel[1]{\hspace\leftmargin$#1$}
% 
\include{hype.indonesia}

\renewcommand{\contentsname}{Daftar Isi}
\renewcommand{\chaptername}{BAB}
\renewcommand{\bibname}{Daftar Referensi}
\renewcommand{\listfigurename}{Daftar Gambar}
\renewcommand\lstlistlistingname{Daftar Program}
\renewcommand{\figurename}{Gambar}
\renewcommand{\tablename}{Tabel}
%\title{Lampiran II}
%\title{Kajian Komputasi Dinamika Fluida berbasis OpenFOAM}
%\author{Arya Adhyaksa Waskita}
%\date{January 31, 2017}
\begin{document}
\include{sampul}
%\tableofcontents

\pagenumbering{roman}
%\maketitle
\clearpage
\setcounter{page}{2}
\addChapter{Daftar Gambar}
\tableofcontents
%\clearpage
\listoffigures
\addChapter{Daftar Program}
\lstlistoflistings
%\clearpage
\pagenumbering{arabic}

\chapter{Pendahuluan}
BATAN saat ini tengah berencana membangun reaktor riset baru berbasis HTGR (\textit{High Temperature Gas-cooled Reactor}) \cite{wang2004integrated} sebagai persiapan PLTN, yang akan dibangun di Indonesia di masa depan \cite{rde}. Salah satu yang perlu diperhatikan dalam pengembangan reaktor jenis ini adalah bahan bakarnya yang berjenis \textit{pebble} yang bentuknya dapat diilustrasikan seperti pada Gambar \ref{fig:bentukpebble}. Bahan bakar harus dirancang sedemikian rupa sehingga rasio gagalnya bahan bakar selama operasi minimal. 

\begin{figure}[h]
  \centering
  \includegraphics[scale=.5]{pics/triso1.png}
  \caption[Ilustrasi bentuk bahan bakar \textit{pebble}]{Ilustrasi bentuk bahan bakar \textit{pebble} \cite{wang2004integrated}}
  \label{fig:bentukpebble}
\end{figure} 

Bahan bakar berjenis \textit{pebble} ini memiliki komponen utama yang dalam Gambar \ref{fig:bentukpebble} disebut sebagai \textit{coated particle}. Komposisi elemen pelapis (\textit{coated}) dapat diilustrasikan dalam Gambar \ref{fig:pelapis}. Dalam upaya menguasai teknologi reaktor berjenis HTGR melalui pengembangan RDE, salah tugas yang harus dilaksanakan adalah penguasaan analisis kegagalan bahan bakarnya, khususnya ketika terjadi kecelakaan.

Beragam model analisis telah dikembangkan, salah satunya yang dikembangkan oleh Wang \cite{wang2004integrated}. Selain itu, terdapat sebuah model sederhana yang dikembangkan oleh Verfondern dalam PANAMA \cite{VERFONDERN201484}. Pada model tersebut, bahan bakar disebut gagal jika kekuatan lapisan SiC (\textit{Silicon Carbide}) lebih kecil daripada tekanan internal dari lapisan di bawahnya (perhatikan Gambar \ref{fig:pelapis}). Model inilah yang akan diterapkan dalam TRIAC (\textit{TRIso Analysis Code}).

\begin{figure}[h]
  \centering
  \includegraphics[scale=.5]{pics/triso.png}
  \caption[Komposisi elemen pelapis partikel]{Komposisi elemen pelapis partikel \cite{wang2004integrated}}
  \label{fig:pelapis}
\end{figure}

\chapter{Alur Perhitungan}
\section{Pendahuluan}
Secara umum, perhitungan TRIAC mengikuti diagram alir seperti pada Gambar \ref{fig:flowchart} berikut. Sementara kode sumbernya disajikan dalam Listing \ref{triac.py} yang dibangun sepenuhnya berbasis pengetahuan yang diperoleh dari dokumen laporan teknis \cite{report1}.
\begin{figure}[h]
  \centering
  \includegraphics[scale=.5]{pics/Flowchart.png}
  \caption{Diagram alir perhitungan TRIAC}
  \label{fig:flowchart}
\end{figure}

\section{Membaca \textit{file input}}
Sub rutin ini ditujukan untuk membaca file input dengan format seperti terdapat pada Lampiran \ref{lamp:inputExample}. Sub rutin ini menggunakan skema yang kaku karena identifikasi nilai-nilai yang akan dibaca ditentukan oleh suatu teks tertentu. Setelah teks yang menjadi penanda, nilai-nilai yang dibutuhkan dibaca. Tetapi, nilai tersebut dapat langsung berada dalam satu baris bersama dengan teks penanda, atau berada pada baris yang berbeda. Sub rutin ini terdapat pada Listing \ref{InputData.py} dan akan dijelaskan pada sub bab \ref{sec:introPenerapan}.
 
\section{Menghitung OPF saat irradiasi}
\label{sec:OPF}
OPF (\textit{Oxygen Per Fission}) adalah jumlah atom oksigen yang terlepas selama fisi atom $U^{235}$ atau $Pu^{239}$. Atom oksigen ini mempengaruhi terbentuknya senyawa CO yang akan meningkatkan tekanan internal dalam bahan bakar. Pembentukan senyawa CO juga dipengaruhi oleh temperatur, waktu serta jenis partikel kernel. 

Nilai OPF didekati oleh persamaan (\ref{eq:opf1}). Nilai $n$ dalam persamaan (\ref{eq:opf1}) sama dengan banyaknya data sejarah irradiasi. Nilai $\Delta_{i}$ merupakan selisih waktu dari sejarah irradiasi yang dicatat. Nilainya akan berubah dengan berubahnya rentang pencatatan temperatur irradiasi. Jika dalam contoh kasus yang disajikan pada Lampiran 1, rentang waktu pencatatan temperatur irradiasi dilakukan setiap $17$ hari, maka $\Delta_{i}$ adalah $17$ hari atau $17 x 24 x 3600$ detik. $t_B$ adalah waktu irradiasi total bahan bakar, sedangkan $\overline{t_i}$ waktu irradiasi ketika pencatatan dilakukan.

\begin{equation}
  OPF \simeq \sum_{i=1}^{n} g(\overline{t_{i}}) \cdot (t_{B}-\overline{t_{i}}) \cdot \Delta t_{i}
  \label{eq:opf1}
\end{equation}
 
Tetapi, nilai OPF juga didefinisikan seperti persamaan (\ref{eq:opf2}), dengan nilai $g(\overline{t_{i}})$ didefinisikan oleh persamaan (\ref{eq:gt}). Nilai $R$ pada persamaan (\ref{eq:gt}) adalah konstanta gas sebesar $8.3143 [\frac{J}{mole \cdot K}]$.
\begin{equation}
  OPF = \frac{g(T)}{2} \cdot t^2
  \label{eq:opf2}
\end{equation} 
 
\begin{equation}
  \frac{g(T)}{2}=8.32 \cdot 10^{-11} \cdot e^{\frac{-163000}{R \cdot T}}
  \label{eq:gt}
\end{equation} 
 
Nilai OPF selanjutnya digunakan untuk menghitung nilai temperatur irradiasi ($T_B$) dari persamaan (\ref{eq:TB}). Formula empiris tersebut sesuai untuk jenis bahan bakar $UO_2$.
\begin{equation}
  \log OPF=-10.08-\frac{0.85 \cdot 10^4}{T_B} + 2 \cdot \log t_B
  \label{eq:TB}
\end{equation} 

Sedangkan nilai $T_B$ akan digunakan untuk menghitung $DS$, faktor berkurangnya koefisien difusi ($s^{-1}$) dari gas hasil fisi di dalam partikel kernel. Nilainya untuk bahan bakar $UO_2$ memenuhi persamaan (\ref{eq:DS}).

\begin{equation}
  \log DS=-2.30-\frac{0.8116 \cdot 10^4}{T_B}
  \label{eq:DS}.
\end{equation}

Terakhir, $DS$ akan digunakan untuk menghitung sebuah nilai tak berdimensi $\tau_i$ yang memenuhi persamaan (\ref{eq:taui}).
\begin{equation}
  \tau_i=DS(T_B) \cdot t_B
  \label{eq:taui}
\end{equation}

\section{Menghitung DS saat kecelakaan}
Seperti telah dijelaskan dalam sub bab \ref{sec:OPF}, $DS$ adalah faktor berkurangnya koefisien difusi gas hasil fisi dalam partikel kernel. Sekarang, faktor ini dihitung ketika kondisi kecelakaan terjadi. Kita memerlukan sejarah temperatur bahan bakar setelah kecelakaan terjadi serta $\tau_i$. yang telah dihitung di persamaan (\ref{eq:taui}).

Dengan menggunakan persamaan (\ref{eq:taui}), kita dapat menghitung nilai $DS$ dengan temperatur kecelakaan yang tercatat.  Kemudian, kita perlu menghitung nilai $\tau_A$ dengan persamaan (\ref{eq:taui}) tetapi dengan nilai temperatur dan waktu setelah terjadi kecelakaan. Selanjutnya, dengan modal nilai $\tau_i$ dan $\tau_A$ kita akan menghitung nilai $Fd$, yang merupakan faktor fisi gas Xe dan Kr (yang dominan). Nilai $Fd$ dihitung dengan persamaan (\ref{eq:Fd}).
\begin{equation}
  Fd=\frac{(\tau_i + \tau_A) \cdot f(\tau_i + \tau_A) - \tau_A \cdot f(\tau_A)}{\tau_i}
  \label{eq:Fd}
\end{equation}

Sedangkan nilai $f(\tau)$ dihitung menggunakan persaamaan (\ref{eq:ftau}). Batas atas nilai $n$ pada persaamaan (\ref{eq:ftau}) dapat menggunakan nilai yang cukup besar, misalnya $1000$, atau ketika dua nilai berdekatan yang dihasilkan hanya berselisih kurang dari $10^{-20}$. Idealnya, suku penjumlahan sebanyak $n$ akan semakin baik jika hasilnya mendekati $1$.
\begin{equation}
  f(\tau)=1-\frac{6}{\tau} \cdot \sum_{n=1}^{\infty} \left( \frac{1-e^{-n^2 \cdot \pi^2 \cdot \tau}}{n^4 \cdot \pi^4} \right)
  \label{eq:ftau}
\end{equation}

%Di level implementasi, perhitungan $DS$ saat kecelakaan didistribusi ke dalam beberapa fungsi seperti terlihat pada Listing \ref{triac.py}. Masing-masing fungsi tersebut adalah \texttt{DS(accident,tauI)} dan \texttt{FD(tauA,tauI)}.

\section{Menghitung tekanan}
Tekanan adalah variabel yang penting dalam tahapan analisis ini karena akan menentukan fraksi gagal bahan bakar. PANAMA \cite{report1} memodelkan fraksi gagal partikel bahan bakar dari sejauh mana lapisan silikon karbida mampu menahan tekanan akibat rilisnya gas produk fisi. Untuk menghitung tekanan yang timbul ketika kecelakaan terjadi pada waktu tertentu, sehingga menyebabkan panas tertentu, digunakan persamaan (\ref{eq:tekanan}) \cite{report1}. 
\begin{equation}
  p=\frac{(F_d \cdot F_f + OPF) \cdot F_b \cdot (\frac{V_k}{V_m}) \cdot R \cdot T}{ V_f}
  \label{eq:tekanan}
\end{equation}
dengan :
\begin{description}
  \item $F_d$ = fraksi relatif gas fisi yang lepas 
  \item $F_f$ = produk fisi yang dihasilkan dari gas fisi stabil, $F_f$=0.31
  \item OPF = jumlah atom oksigen setiap terjadi fisi saat terjadi kecelakaan
  \item $F_b$ = \textit{burnup} logam berat (FIMA)
  \item $V_f$ = fraksi void [$m^3$], terkait dengan $50\%$ volume buffer 
  \item $V_k$ = volume kernel [$m^3$]
  \item $V_m$ = volume molar dalam partikel kernel $\left[\frac{m^3}{mole} \right]$, didefinisikan sebagai rasio berat 1 mol material kernel terhadap kerapatannya. Menurut Verfondern \cite{report1}, $V_m$ untuk $(Th,U)O_2$, $UO_2$ dan UCO masing-masing adalah $2.52 \cdot 10^{-5}[\frac{m^3}{mole}]$, $2.44 \cdot 10^{-5}[\frac{m^3}{mole}]$ dan $2.51 \cdot 10^{-5}[\frac{m^3}{mole}]$.
  \item $R$ = konstanta gas, 8.3143 $\left[\frac{J}{(mole \cdot K)} \right]$
\end{description}

Khusus untuk variabel $OPF$, karena perhitungan tekanan dilakukan ketika terjadi kecelakaan, digunakanlah persamaan (\ref{eq:opf3}). Persamaan (\ref{eq:opf3}) mirip dengan persamaan (\ref{eq:TB}) dengan penambahan suku ke-3.
\begin{equation}
  \log OPF=-10.08-\frac{0.85 \cdot 10^4}{T_B} + 2 \cdot \log t_B - 0.04 \cdot \left( \frac{10^4}{T} + \frac{10^4}{T_B + 75} \right)
  \label{eq:opf3}
\end{equation}

\section{Fraksi gagal bahan bakar}
Tahapan terkahir dari analisis ini adalah perhitungan fraksi gagal bahan bakar. Secara umum, fraksi gagal bahan bakar dipengaruhi sejumlah sebab. Dalam analisis yang dilakukan TRIAC (dan juga PANAMA sebagai acuannya), gagalnya bahan bakar dapat disebabkan oleh 3 sebab. Ketiganya adalah sebagai berikut.
\begin{enumerate}
\item Pabrikasi ($\phi_0$). Dalam analisis ini, nilai $\phi_0$ diasumsikan sama dengan $0$.
\item Berkurangnya \textit{tensile strength} lapisan SiC ($\phi_1$). Hal ini dapat terjadi karena
\begin{itemize}
\item proses irradiasi maupun
\item meningkatnya temperatur secara signifikan ketika terjadi kecelakaan) atau disebut juga \textit{grain boundary}.
\end{itemize}  
\item Dekomposisi termal pada temperatur tinggi yang menyebabkan terjadinya \textit{weight loss} pada lapisan SiC ($\phi_2$).
\end{enumerate}

Ketiga sebab terjadinya kegagalan bahan bakar tersebut mengikuti persamaan (\ref{eq:gagal1}).
\begin{equation}
  \phi_{total}=1-(1-\phi_0) \cdot(1-\phi_1) \cdot (1-\phi_2)
  \label{eq:gagal1}
\end{equation}

\subsection{Fraksi gagal akibat berkurangnya \textit{tensile strength}}
Fraksi gagal partikel triso dimodelkan dengan apa yang diistilahkan Verfondern sebagai model bejana tekan \cite{report1}. Hal ini disebabkan karena  fraksi gagal dipengaruhi oleh variabel-variabel yang terenkapsulasi dalam parameter tenanan internal dan kekuatan lapisan silikon karbida. Nilai fraksi gagal bahan bakar pada waktu $t$ setelah terjadinya kecelakaan diperoleh dengan persamaan (\ref{eq:gagal}).
\begin{equation}
  \phi_1(t,T)=1-e^{-\ln 2 \cdot \left(\frac{\sigma_t}{\sigma_o}\right)^m}
  \label{eq:gagal}
\end{equation}
dengan :
\begin{description}
  \item $\sigma_o$=\textit{tensile strength} dari SiC [Pa] pada akhir irradiasi
  \item $\sigma_t$=tekanan yang dialami SiC [Pa] akibat tekanan gas internal
  \item $m$=parameter Weibull (dijelaskan selanjutnya)
\end{description}

Variabel tekanan internal pada SiC ($\sigma_t$) dihitung dengan dengan persamaan (\ref{eq:sigmaT}). Pada persamaan (\ref{eq:sigmaT}), jari-jari lapisan SiC merupakan rerata karena lapisan SiC memang memiliki ketebalan yang nilai awalnya diwakili oleh variabel $d_o$.
\begin{equation}
  \sigma_t =\frac{r \cdot p}{2 \cdot d_o} \cdot \left( 1+\frac{\dot{v} \cdot t}{d_o} \right)
  \label{eq:sigmaT}
\end{equation}
dengan :
\begin{description}
  \item $r$=rerata jari-jari SiC, $\left( 0.5 \cdot \left( r_a^3 + r_i^3 \right)\right)^{\frac{1}{3}}$ [m]
  \item $d_o$=ketebalan awal lapisan SiC, $r_a - r_i$ [m]
  \item $p$=tekanan gas fisi dalam partikel [Pa], dihitung menggunakan persamaan (\ref{eq:tekanan})
  \item $\dot{v}$=laju korosi sebagai fungsi temperatur (T), $\left[\frac{m}{s}\right]$
\end{description}

Sedangkan variabel laju korosi ($\dot{v}$) dihitung dengan persamaan (\ref{eq:korosi}), mirip dengan persamaan (\ref{eq:gt}) dengan perbedaan pada konstanta.
\begin{equation}
  \dot{v}=5.87 \cdot 10^{-7} \cdot e^{-\left( \frac{179500}{R \cdot T}\right)}
  \label{eq:korosi}
\end{equation}

Selanjutnya, variabel \textit{tensile strength} lapisan SiC, penurunan nilainya mengikuti persamaan (\ref{eq:strengthSiC}). Variabel $\sigma_{oo}$ merupakan \textit{tensile strength} awal sebelum diiradiasi. Nilainya merupakan sesuatu yang dapat diukur. Sedangkan $\Gamma$ dan $\Gamma_s$ masing-masing merupakan \textit{fluence} netron cepat $\left[ 10^{25}m^{-2} EDN\right]$ dan \textit{fluence} yang dipengaruhi temperatur irradiasi. Nilai $\Gamma_s$ ditentukan menggunakan persamaan (\ref{eq:fluenceS}). %Nilai minimum $\sigma_{oo}$ merupakan nilai awal \textit{tensile strength} dan diasumsikan sama dengan 196 [MPa]. Tentunya, dengan perlakuan irradiasi yang sama, lapisan SiC dengan nilai awal \textit{tensile strength} terkecil akan memiliki nilai akhir \textit{tensile strength} yang juga kecil.
\begin{equation}
  \sigma_o = \sigma_{oo} \cdot \left( 1- \frac{\Gamma}{\Gamma_s} \right)
  \label{eq:strengthSiC}
\end{equation}

\begin{equation}
  \log \Gamma_s = 0.556 + \frac{0.065 \cdot 10^4}{T_B}
  \label{eq:fluenceS}
\end{equation}

\textit{Tensile strength} lapisan SiC yang dihitung menggunakan persamaan (\ref{eq:strengthSiC}) merupakan nilai yang berlaku pada satu \textit{coated particle}. Padahal, ada sangat banyak \textit{coated particle} yang dioperasikan. Karena itu, diperlukan perhitungan yang mempertimbangkan variabel ini untuk semua distribusi \textit{coated particles}. Dengan pendekatan yang sama seperti persamaan (\ref{eq:strengthSiC}), persamaan (\ref{eq:strengthSiCdist}) dibangun. Nilai $\Gamma_m$ ditentukan menggunakan persamaan (\ref{eq:fluenceM}).

\begin{equation}
  m_o = m_{oo} \cdot \left( 1- \frac{\Gamma}{\Gamma_m} \right)
  \label{eq:strengthSiCdist}
\end{equation}

\begin{equation}
\log \Gamma_m = 0.394 + \frac{0.065 \cdot 10^4}{T_B}
\label{eq:fluenceM}
\end{equation}

%Nilai $m_o$ pada persamaan (\ref{eq:strengthSiCdist}) kemudian akan disubstitusi ke persamaan (\ref{eq:gagal}) sebagai $m$. Pada konteks ini, parameter $m$ yang merupakan salah satu parameter dalam persamaan (\ref{eq:gagal}) ditentukan berdasarkan kekuatan (\textit{tensile strength}) lapisan SiC. $m$ menentukan distribusi kekuatan SiC pada sejumlah partikel TRISO. 

Sama seperti $\sigma_{oo}$, nilai $m_{oo}$ juga diperoleh dengan mengukur parameter tersebut pada partikel yang belum diiradiasi. Tabel \ref{tab:oo} menunjukkan nilai $\sigma_{oo}$ dan $m_{oo}$ pada beberapa jenis specimen sebelum dikenakan irradiasi \cite{report1}.%Nilainya akan berkurang sejalan dengan proses irradiasi serta mengikuti persamaan (\ref{eq:gagal}). 


\begin{table}
  \caption[Nilai $\sigma_{oo}$ dan $m_{oo}$ untuk berbagai jenis specimen]{Nilai $\sigma_{oo}$ dan $m_{oo}$ untuk berbagai jenis specimen\cite{report1}}
  \label{tab:oo}

  \begin{center}
  \ra{1.3}
    \begin{tabular}{@{}lrrcrr@{}}\toprule
    %\hline
    Specimen & \multicolumn{2}{c}{Sebelum irradiasi} & \phantom{abc} & \multicolumn{2}{c}{Setelah irradiasi} \\ \cmidrule{2-3} \cmidrule{5-6} 
       & $\sigma_{oo}$ [MPa] & $m_{oo}$ && $\sigma_{o}$ [MPa] & $m_{o}$ \\ \midrule%\hline
       EO 1674 & 722 & 7.0 && 660 & 6.1\\
       EO 1607 & 850 & 8.0 && 777 & 7.0\\
       HT 150-167 & 600 & 6.0 && 549 & 5.3\\
       EO 249-251 & 453 & 5.0 && 414 & 4.4\\
       EO 403-405 & 867 & 8.4 && 793 & 7.4\\
       EUO 1551 & 1060 & 8.5 && 969 & 7.4\\
       ECO 1541 & 1080 & 6.4 && 987 & 5.6\\
       EC 1338 & 998 & 7.4 && 912 & 6.5\\ %\hline 
       \bottomrule
    \end{tabular}
  \end{center}
\end{table}


Selain korosi karena proses irradiasi, lapisan SiC juga dapat terkorosi karena \textit{grain Boundary}. Jika korosi akibat irradiasi tergantung pada sejarah irradiasi yang dialami bahan bakar dan terjadi sebelum kecelakaan, maka korosi karena \textit{grain Boundary} terjadi setelah kecelakaan. Penurunan nilai distribusi \textit{tensile strength} akibat meningkatnya temperatur karena kecelakaan mengikuti persamaan (\ref{eq:grainBoundary}), di mana nilai $m_o$ diperoleh dari persamaan (\ref{eq:strengthSiCdist})
\begin{equation}
  m=m_o \cdot \left( 0.44 + 0.56 \cdot e^{-\dot{\eta} \cdot t}\right)
  \label{eq:grainBoundary}
\end{equation}

dan nilai $\dot{\eta}$ mengikuti persamaan {\ref{eq:eta}} dengan pola yang sama seperti persamaan (\ref{eq:korosi}).
\begin{equation}
  \dot{\eta}=0.565 \cdot e^{\left(\frac{-187400}{R \cdot T}\right)} [s^{-1}]
  \label{eq:eta}
\end{equation}

\subsection{Fraksi gagal bahan bakar akibat \textit{weight loss}}
Laju \textit{weight loss} yang terjadi akibat tingginya temperatur saat terjadi kecelakaan mengikuti persamaan (\ref{eq:weightLoss}).
\begin{equation}
  k=k_o \cdot e^{\frac{-Q}{R \cdot T}}
  \label{eq:weightLoss}
\end{equation}
dengan $Q=556 \left[ \frac{kJ}{mol}\right]$ dan $k_o$ adalah faktor frekuensi yang tergantung pada jenis partikel.

Selanjutnya, diasumsikan bahwa partikel TRISO tergantung pada apa yang disebut sebagai ''\textit{action integral}'', dan disimbolkan dengan $\zeta$ yang nilainya mengikuti persamaan (\ref{eq:zeta}).
\begin{equation}
  \zeta=\int_{t_1}^{t_2} k(T) dt
  \label{eq:zeta}
\end{equation}
dengan $K(T)$ adalah nilai yang menggambarkan sejarah kondisi partikel yang bergantung pada temperatur dan waktu.

Secara numerik, persamaan (\ref{eq:zeta}) dapat dituliskan sebagai persamaa (\ref{eq:zeta1}).
\begin{equation}
  \zeta(t_2)=\zeta(t_1)+k(T_m) \cdot (t_2 - t_1)
  \label{eq:zeta1}
\end{equation}
dengan $k(T_m)=\frac{375}{d_o} \cdot e^{\left(\frac{-556000}{R \cdot T_m}\right)}$.

Kemudian, fraksi gagal $\phi_2$ sedemikian rupa sehingga nilainya $\leq 1$. Karena itu, variabel $\phi_2$ selanjutnya didefinisikan sebagai persamaan (\ref{eq:phi2}).
\begin{equation}
\phi_2(t,T)=1-e^{-\alpha \cdot \zeta^{\beta}}
\label{eq:phi2}
\end{equation}

Nilai $\alpha$ dan $\beta$ kemudian ditentukan secara empiris. Dan berdasarkan penelitian empiris sebelumnya terhadap partikel $UO_2$, diperoleh nilai $\alpha=\ln 2=0.693$, sedangkan nilai $\beta=0.88$.

Dalam TRIAC, faktor fraksi gagal ini tidak akan dipertimbangkan. Hal ini disebabkan karena kondisi ini terjadi pada temperatur di atas $2000\,^{\circ}{\rm C}$. Sementara RDE tidak dirancang untuk sampai pada temperatur tersebut.

\subsection{Pertumbuhan fraksi gagal}
Berdasarkan PANAMA \cite{report1}, Verfondern memodelkan pertumbuhan fraksi gagal partikel triso akibat berkurangnya \textit{tensile strength} adalah seperti persamaan (\ref{eq:fraksigagal}) berikut.
\begin{equation}
\phi_1=\phi_1(t_2,T_m)-\phi_1(t_1,T_m)
\label{eq:fraksigagal}
\end{equation}

$T_m$ merupakan temperatur rata-rata antara waktu $t_1$ dan $t_2$. Ilustrasinya disajikan dalam Gambar \ref{fig:pertumbuhanPhi} 
\begin{figure}
  \begin{center}
    \includegraphics[scale=.5]{pics/akumulasiPhi.png}
    \caption{Hubungan antara waktu dan temperatur pada perhitungan $\phi_1$}
    \label{fig:pertumbuhanPhi}
  \end{center}
\end{figure}

Disebutkan Verfondern \cite{report1}, nilai $\phi_1$ saat kecelakaan dimulai ($t_0$ seperti pada Gambar \ref{fig:pertumbuhanPhi}) merupakan fungsi dari sejarah irradiasi. Sedangkan untuk waktu-waktu selanjutnya ($t_1, t_2, \cdots, t_n$) merupakan akumulasi dari nilai $\phi_1$ pada persamaan (\ref{eq:fraksigagal}). Jika $\phi_1$ pada $t_1$ lebih besar daripada $\phi_1$ pada saat $t_0$, maka akumulasikan nilai $\phi_1$. Jika sebaliknya, gunakan nilai $\phi_1$ sebelumnya untuk perhitungan selanjutnya (nilai $\phi_1$ tetap). 

Sebagai ilustrasi, saat menghitung nilai $\phi_1$ di $t=t_1$, maka diperlukan nilai $\phi_1(t_0, T_m)$ (nilai pertama) dan $\phi_1(t_1, T_m)$ (nilai kedua). Nilai pertama adalah fungsi irradiasi, sedangkan nilai kedua diperoleh dari persamaan (\ref{eq:gagal}) dengan parameter-parameter yang sesuai. Selisih keduanya akan menentukan nilai $\phi_1$ di titik $t=t_1$. Jika selisih nilai kedua dan pertama positif, selisih nilai tersebut diakumulasikan pada nilai $\phi_1$ di $t=t_0$. Tetapi jika sebaliknya, maka nilai $\phi_1$ di titik $t=t_1$ sama dengan nilai $\phi_1$ di titik $t=t_0$. Skenario yang sama berlaku untuk titik-titik waktu selanjutnya.

\chapter{Penerapan}
\section{Pendahuluan}
\label{sec:introPenerapan}
TRIA \textit{Code} yang telah dijelaskan sebelumnya secara umum dapat dikelompokkan menjadi dua tugas utama, masing-masing adalah perhitungan di waktu irradiasi dan kecelakaan. Saat irradiasi, hubungan saling ketergantungan antar variabel adalah seperti Gambar \ref{fig:irradiasi}. Sedangkan saat kecelakaan, hubungannya adalah seperti pada Gambar \ref{fig:accident}.

\begin{figure}[h]
  \begin{center}
    \includegraphics[scale=.5]{pics/alurFormula2.png}
    \caption{Hubungan ketergantungan antar variabel di fase irradiasi}
    \label{fig:irradiasi}
  \end{center}
\end{figure}

\begin{figure}[h]
  \begin{center}
    \includegraphics[scale=.5]{pics/alurFormula.png}
    \caption{Hubungan ketergantungan antar variabel di fase kecelakaan}
    \label{fig:accident}
  \end{center}
\end{figure}

Selanjutnya, triac juga memerlukan sejumlah data yang harus diberikan oleh pengguna sebelum perhitungan dimulai. Selain data-data seperti yang akan dijelaskan dalam sub bab \ref{sec:fileinput}, diperlukan juga beberapa data lain. Karena triac mengadopsi perhitungan yang dilakukan dalam PANAMA \cite{report1}, maka triac juga memerlukan data seperti yang diperlukan PANAMA. Tabel \ref{tab:additionalData} menyajikan beberapa parameter serta nilainya yang diperlukan oleh triac, masing untuk HTR-Modul dan HTR-500.

\begin{table}[h]
  \caption[Tambahan data yang diperlukan triac]{Tambahan data yang diperlukan triac\cite{report1}}
  \label{tab:additionalData}

  \begin{center}
  \ra{1.3}
    \begin{tabular}{@{}lrr@{}}\toprule
    %\hline
    Parameter & HTR-Modul & HTR-500 \\ \midrule%\hline
       Jenis partikel & $UO_2$ & $UO_2$ \\ 
       \textit{Burnup} / FIMA / $Fb$ & 0.08 & 0.08 \\
       \textit{Fast fluence} / $\Gamma$ [$10^{25}m^{-2}$] & 1.4 & 1.4 \\
       $\sigma_{oo}$ [MPa] & 834 & 834 \\
       $m_{oo}$ & 8.02 & 8.02 \\
       \bottomrule
    \end{tabular}
  \end{center}
\end{table}

Selain itu, triac juga memerlukan parameter lain berupa status interpolasi. Dengan status ini, sejarah irradiasi/kecelakaan akan diinterpolasi atau menggunakan nilai yang diberikan pengguna dari \textit{file input}.

\section{Pembacaan \textit{file input}}
\label{sec:fileinput}
Seluruh proses dalam triac didahului dengan membaca \textit{file input} dengan format yang sama seperti pada Lampiran \ref{lamp:inputExample}. Penerapan pembacaan \textit{file input} adalah seperti pada Listing \ref{InputData.py}.%\ref{triac.py}.

Di Listing \ref{InputData.py}, pembacaan \textit{input data} dilakukan secara sekuensial dan manual. Nilai-nilai yang harus dibaca ditentukan berdasarkan informasi yang ada pada \textit{file input}. Sebagai contoh, untuk membaca nilai geometri, digunakan karakter ''[m]'' sebagai penanda. Jika ditemukan karakter tersebut, maka di saat itulah pembacaan nilai geometri dilakukan. Hal inilah yang dimaksud sebagai pembacaan secara manual. Ketika karakter yang diperlukan berubah, maka modifikasi harus dilakukan pada modul ini.

Selain nilai terkait geometri, diperlukan juga pembacaan untuk nilai \textit{physical properties} serta sejarah operasi, baik saat operasi normal maupun kecelakaan. Pembacaan nilai yang berbeda dilakukan secara berurutan berdasarkan kemunculan nilai tersebut dalam \textit{file input}. Hal inilah yang dimaksud dengan pembacaan secara sekuensial.

Terdapat empat jenis data yang perlu dibaca dari \textit{file input} dalam Lampiran 1, masing-masing adalah sebagai berikut. Penerapannya disajikan dalam Listing \ref{InputData.py}.
\begin{enumerate}
  \item Data tentang geometri \textit{pebble}. Data ini diidentifikasi menggunakan teks yang didefinisikan oleh variabel \texttt{statusGeometry} (baris ke-4. Di dalam data geometri, terdapat empat data berbeda, masing-masing secara berurutan adalah panjang jejari \textit{pebble} terluar, OPyC (\textit{Outer Pyrolitic Carbon}), SiC (\textit{Silicon Carbide}), IPyC (\textit{Inner Pyrolitic Carbon}), \textit{buffer} dan kernel. Data geometri akan digunakan untuk menghitung volume setiap elemen pelapis (Gambar \ref{fig:pelapis}). Yang perlu diperhatikan adalah data jari-jari yang disajikan adalah jarak dari pusat bahan bakar sampai titik terluar dari setiap lapisan. Karena itu, volume suatu lapisan harus mempertimbangkan lapisan-lapisan di dalamnya. Data geometri disimpan dalam variabel diberi nama \texttt{dimensi} dan dalam bentuk \texttt{list} (baris ke-9).
  \item Data tentang kekuatan SiC. Data ini diidentifikasi menggunakan teks yang didefinisikan oleh variabel \texttt{statusCharacteristics} (baris ke-5 pada Listing \ref{InputData.py}). Ada empat nilai yang perlu dibaca terkait kekuatan SiC, masing-masing adalah SiC \textit{Tensile Strength} [Pa],	\textit{Weibull Modulus	Burnup} [FIMA],	\textit{Fission Yield of stable fission gasses} [Ff],	\textit{Fast Neutron Fluence}	dan rasio berat Th terhadap U-235 pada kernel. Data terkait kekuatan SiC disimpan dalam variabel yang diberi nama \texttt{characteristics} dalam bentuk \texttt{list} (baris ke-10).
    \item Data tentang sejarah irradiasi. Data ini diidentifikasi menggunakan teks yang didefinisikan oleh variabel \texttt{statusIrradiation} (baris ke-6 pada Listing \ref{InputData.py}). Data ini merupakan data temperatur bahan bakar \textit{pebble} pada selang waktu tertentu. Sebagai contoh, data yang disajikan pada Lampiran 1 diambil pada selang waktu 17 hari. Data sejarah irradiasi disimpan dalam variabel yang diberi nama \texttt{irradiation} dalam bentuk \texttt{list}. Setiap elemen adalah \texttt{list} yang secara \textit{nested} terdiri dari dua elemen yang mewakili data kolom kedua dan ketiga tiap akuisisi (baris ke-11). Ilustrasinya adalah seperti $[[0,	593], [1468800,	833], \ldots]$ dengan informasi waktu pengukuran dalam satuan detik. Data tentang nomor urut tidak digunakan karena selain tidak diperlukan dalam perhitungan, akan menyulitkan proses interpolasi yang akan diterapkan berikutnya.
    \item Data tentang sejarah kecelakaan. Data ini diidentifikasi menggunakan teks yang didefinisikan oleh variabel \texttt{statusAccident} (baris ke-7). Data ini memiliki pola yang sama dengan data sejarah irradiasi. Data sejarah keselakaan disimpan dengan cara yang sama seperti data tentang sejarah irradiasi tetapi dengan nama \texttt{accident} (baris ke-12). Ilustrasinya adalah seperti $[[0,	1033], [2341.44,	1033],\ldots]$ dengan informasi waktu pengukuran dalam satuan detik.
\end{enumerate}.

Namun, terlihat pada baris ke-76 dari Listing \ref{InputData.py}, terdapat total 5 variabel yang dikembalikan ke fungsi awal, dengan variabel kelima adalah $b-a$. Variabel ini adalah rentang waktu pengukuran data irradiasi.


Selain itu, untuk meningkatkan ketelitian perhitungan, disiapkan juga modul interpolasi secara linier. Modul ini disiapkan agar sejarah operasi normal dan kecelakaan sehingga dapat diperoleh hasil yang tepat. Penerapan dari modul interpolasi linier tersebut disajikan pada Listing \ref{Interpolasi.py}.

Seperti terlihat pada Lampiran \ref{lamp:inputExample}, sejarah operasi normal atau disebut juga sebagai sejarah irradiasi, terdapat 3 kolom dalam \textit{file input}. Demikian juga untuk sejarah ketika terjadi kecelakaan. Ketiganya adalah nomor urut, hari ke sekian dan temperatur. Dengan melakukan interpolasi, selisih hari yang digunakan dapat diperkecil. Dalam contoh \textit{file input}, selisih pencatatan adalah 17 hari. Dengan interpolasi, kita dapat mengestimasi sejarah dalam selisih waktu yang lebih singkat. 

Interpolasi yang diterapkan dapat diilustrasikan dalam Gambar \ref{fig:interpolasi} \footnote{\url{http://jadipaham.com/wp-content/uploads/2015/10/Rumus-interpolasi-linear.jpg}}. Argumen ketiga dari fungsi \texttt{linier (a,b,c)}, $c$, adalah jumlah partisi diantara nilai $x_1$ dan $x_2$. Nilai tersebut adalah $dt$ yang merupakan argumen ketika mengeksekusi kode komputer TRIAC (Listing \ref{triac.py}). Penggunaan fungsi interpolasi ini dilakukan di Listing \ref{triac.py} pada baris ke-53 s/d 66.

\begin{figure}
  \centering
  \includegraphics[scale=.5]{pics/interpolasi-linear.png}
  \caption{Ilustrasi interpolasi linier yang digunakan}
  \label{fig:interpolasi}
\end{figure}


\section{TRIAC Core}
Terdapat empat fungsi di dalam modul \texttt{core.py} seperti terlihat pada Listing \ref{core.py}. Fungsi-fungsi yang terdapat dalam modul ini dianggap sebagai fungsi yang sering digunakan dan relatif kompleks jika diletakkan dalam program utama TRIAC. Berikut adalah penjelasannya.

\begin{enumerate}
\item \texttt{OPF (irradiation,y,tb)} (baris ke-3 s/d 15). Fungsi tersebut membutuhkan tiga argumen, dengan argumen pertama adalah sejarah irradiasi dalam bentuk array. Jika melihat contoh yang disajikan pada Lampiran \ref{lamp:inputExample}, data tersebut terletak setelah baris berisi \texttt{INPUT: Irradiation Temp. Hystory}. Array akan berdimensi dua, yaitu setiap elemen array merupan array dengan dua elemen, masing-masing adalah waktu (dalam detik) dan temperatur irradiasi.

Argumen kedua, $y$ adalah rentang waktu pengukuran ketika massa irradiasi. Contoh pada Lampiran 1 menunjukkan bahwa pengukuran dilakukan setiap 17 hari. Ketika kita ingin rentang pengukuran ini lebih kecil dari 17, maka kita dapat memperolehnya dengan fungsi interpolasi. Waktu interpolasi ini harus dalam satuan detik. Sedangkan argumen ketiga adalah total masa irradiasi, yang dalam contoh Lampiran 1 adalah 1020 hari.

Fungsi OPF akan menghitung akumulasi nilai $g$ untuk setiap perubahan temperatur dan waktu irradiasi seperti dijelaskan pada persamaan (\ref{eq:gt}). Akumulasi nilai $g$ tersebut adalah nilai OPF seperti dijelaskan pada persamaan (\ref{eq:opf1}). Itu sebabnya kenapa fungsi ini diberi nama \texttt{OPF}.

Kemudian, nilai OPF digunakan untuk menghitung nilai $Tb$ seperti dijelaskan persamaan (\ref{eq:TB}). Nilai $Tb$ selanjutnya digunakan untuk menghitung nilai $ds$ seperti dijelaskan persamaan (\ref{eq:DS}) dan diterapkan oleh fungsi \texttt{DS} (baris ke-31 s/d 34 Listing \ref{core.py}). Akhirnya, nilai $\tau_i$ diperoleh dari nilai $ds$ seperti dijelaskan persamaan (\ref{eq:taui}). Tetapi, rentetan perhitungan tersebut tidak dilakukan di fungsi \texttt{OPF}, melainkan dalam program \texttt{triac.py} seperti pada Listing \ref{triac.py}. Fungsi yang dibuat diusahakan untuk hanya mengerjakan satu fungsi saja.

\item \texttt{FTau (tau)} (baris ke-17 s/d 25). Fungsi ini digunakan untuk menghitung nilai $f_{\tau}$ seperti dijelaskan pada persamaan (\ref{eq:ftau}). Fungsi ini menerapkan nilai $2000$ sebagai batas atas iterasi.

\item \texttt{OPFAccident (Tb,tb,T)} (baris ke-27 s/d 30). Fungsi ini akan menerima argumen kondisi kecelakaan melalui argumen ketiga ($T$). Sedangkan argumen pertama ($Tb$) diperoleh dari fungsi pertama, \texttt{OPF}.

\item \texttt{DS(T)} (baris ke-32 s/d 35). Fungsi ini adalah fungsi perantara untuk mendapatkan nilai $\tau_i$.

\item \texttt{volume(r)}. Fungsi ini digunakan untuk menghitung volume setiap lapisan partikel triso.

\item \texttt{weibullParam(Tb,T,t,m00,gamma)}. Fungsi ini digunakan untuk menghitung parameter $m$ seperti dijelaskan oleh persamaan (\ref{eq:grainBoundary}). Fungsi ini digunakan sejak perhitungan memasuki kondisi kecelakaan. Diperlukan 4 argumen untuk mengeksekusi fungsi ini, masing-masing adalah $T_b$ (diperoleh dari perhitungan terhadap sejarah irradiasi), $T$ dan $t$ (masing-masing adalah temperatur dan waktu sejak terjadinya kecelakaan), $m_{oo}$ (parameter yang menggambarkan distribusi Weibull \textit{tensile strength} untuk semua partikel triso sebelum diirradiasi), serta $\Gamma$.

\item \texttt{tekanan(Fd,opf,Vk,T,Vf)} (baris ke-51 s/d 56). Diperlukan 5 argumen untuk menghitung nilai tekanan yang telah dijelaskan di persamaan (\ref{eq:tekanan}). Khusus untuk argumen $opf$, nilainya diperoleh dari fungsi \texttt{OPFAccident}.

\item \texttt{sigmaT(r,p,d,t,T)} (baris ke-58 s/d 61). Diperlukan 5 argumen untuk menghitung nilai $\sigma_t$. Argumen $r$ dan $d$ adalah parameter yang dijelaskan di persamaan (\ref{eq:sigmaT}). Khusus untuk argumen $d$, di persamaan (\ref{eq:sigmaT}) dinyatakan sebagai $d_0$. Argumen $t$ dan $T$ adalah waktu dan temperatur setelah terjadinya kecelakaan. Khusus untuk argumen $T$, nilainya tidak serta-merta sesuai dengan waktu $t$. Seperti dijelaskan pada Gambar \ref{fig:pertumbuhanPhi}, nilai temperatur yang digunakan dalam perhitungan adalah nilai rata-rata dari dua waktu akuisisi data yang beriringan. Sedangkan argumen $p$ adalah variabel tekanan yang telah dijelaskan sebelumnya.

\item \texttt{phi(sigma0, sigmaT, m)} (baris ke-63 s/d 67). Parameter inilah yang menjadi fokus triac, yaitu memprediksi fraksi gagal partikel triso seperti telah dijelaskan di persamaan (\ref{eq:gagal}). 
\end{enumerate}



\section{Perhitungan TRIAC}
Bagian ini adalah inti dari perhitungan TRIAC yang alur eksekusinya diilustrasikan pada Gambar \ref{fig:flowchart}. Sedangkan hubungan interaksi antar fungsi untuk mendapatkan nilai fraksi gagal partikel triso setelah sekian waktu sejak terjadi kecelakaan dapat diilustrasikan seperti Gambar \ref{fig:interaksiformula}. Kotak dengan warna merah, kuning dan hijau pada Gambar \ref{fig:interaksiformula} menunjukkan formula-formula yang hasilnya menjadi masukan untuk formula pada kotak berwarna biru. Sementara angka di bawah kotak-kotak tersebut adalah nomor formula dalam dokumen PANAMA \cite{report1}.

\begin{figure}[h]
  \begin{center}
    \includegraphics[scale=.5]{pics/alurRumus1.png}
    \caption{Interaksi antar fungsi untuk mendapatkan fraksi gagal partikel triso}
    \label{fig:interaksiformula}
  \end{center}
\end{figure}


Sedangkan Gambar \ref{fig:interaksiformula2} merupakan kelanjutan dari interaksi yang ditunjukkan Gambar \ref{fig:interaksiformula}, khususnya untuk menyediakan nilai masukan bagi parameter nilai tekanan yang dialami lapisan silikon karbida. Sama seperti Gambar \ref{fig:interaksiformula}, angka di bawah kotak-kotak berwarna yang berisi formula pada Gambar \ref{fig:interaksiformula2} menunjukkan nomor formula pada dokumen PANAMA \cite{report1}. Selain informasi nomor persamaan pada dokumen PANAMA, ditunjukkan pula bahwa kotak berwarna kuning merupakan variabel yang dipengaruhi oleh jenis partikel triso. Formula yang disajikan merupakan formula empiris untuk jenis partikel $UO_2$. Sementara untuk kotak berwarna hijau, selain dipengaruhi oleh jenis material partikel triso, juga dipengaruhi oleh kondisi apakah partikel triso sedang berada pada masa irradiasi (formula pertama) atau kecelakaan (formula kedua). 

%Dari Gambar \ref{fig:interaksiformula} dan \ref{fig:interaksiformula2} 

\begin{figure}[h]
  \begin{center}
    \includegraphics[scale=.5]{pics/alurRumus2.png}
    \caption{Interaksi antar fungsi untuk mendapatkan nilai tekanan yang dialami lapisan silikon karbida}
    \label{fig:interaksiformula2}
  \end{center}
\end{figure}

Program ini akan menerima dua argumen selain nama programnya sendiri, yaitu nama \textit{file input} (baris ke-8) serta jumlah interpolasi yang diinginkan (baris ke-9) dalam rentang pengukuran yang sudah ada. Tetapi, jika pengguna tidak memberikan argumen tersebut, program akan dieksekusi dengan \textit{file input} dan jumlah interpolasi yang telah ditetapkan (baris ke-11 dan 12).

Proses selanjutnya adalah membaca informasi dari \textit{file input}. Ada lima informasi yang harus diperoleh dari \textit{file input}, masing-masing adalah
informasi geometri, karakteristik material, sejarah irradiasi dan kecelakaan serta rentang pengukuran temperatur saat irradiasi. Setelah data-data tersebut diperoleh, langkah selanjutnya adalah perhitungan geometri partikel triso. Proses ini dilakukan dalam baris ke-23 s/d 37.

Langkah selanjutnya setelah perhitungan geometri adalah interpolasi. Tetapi, karena opsi tanpa interpolasipun harus diakomodasi, maka ada kondisi yang harus dipenuhi seperti pada baris ke-44 dan baris ke-96. Jika nilai $dt>1$, maka interpolasi harus dilakukan.



% Daftar Pustaka
\bibliographystyle{IEEEtran}
\bibliography{report}

\begin{appendix}
	\include{markLampiran}
	\setcounter{page}{2}
	%-----------------------------------------------------------------------------%
\addChapter{Lampiran 1}
\chapter*{Lampiran 1}
%-----------------------------------------------------------------------------%

\includepdf{inputExample1}
\includepdf{inputExample2}
%\label{lamp:inputExample}

\end{appendix}

\end{document}
